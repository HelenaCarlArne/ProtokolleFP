\section{Diskussion}
\label{sec:Diskussion}

\subsection{Diskussion über das angewandte Verfahren und des Fehlers}
\label{sec:Diskussion1}
Ein Tomograph ist ein wichtiges Instrument, um Struktur- und Materialaufklärung
zu gewährleisten. In der Medizin werden Geräte verwendet, die die Menge an
inkorperierter Strahlung gering halten. Ein $\gamma$-Strahler ist also a priori
ausgeschlossen. In der Materialprüfung, insbesondere bei Metallen, müssen hingegen
Strahlungsquellen einer hohen Intensität benutzt werden, um in vertretbarer Zeit
ausreichend Zerfälle zu registrieren und damit ein verlässiliches Ergebnis zu
erzielen. Das Durchleuchten eines Lkw - als vereinendes Beispiel - verwendet
ernstzunehmende Strahlung, wird aber erst nach Passieren der Fahrerkabine gestartet.

Als poisson-verteilter Zerfallsprozess kann eine Formel gefunden werden, die die
gewünschte Messunsicherheit $p$, hier 3\%, und die dazu erforderliche Anzahl $N$
verknüpft
\begin{equation}
  N=\frac{1}{p^2}\approx 1100
\end{equation}
Der statistische Fehler ist damit als bekannt und vernachlässigbar gegenüber eventuell
auftretenden systematischen Fehlern anzusehen. Diese systematischen Fehler liegen
im Wesentlichen an der Probenausrichtung für eine Projektion, welche die Absorptionskoeffizienten
untereinander mischen lassen. Selbst bei idealer Ausrichtung muss weiter angenommen
werden, dass der kummulierte Strahl divergiert und damit mehr Elementarwürfel als
vorgesehen erfasst.

Die numerische Brauchbarkeit der Projektionsmatrix wurde überprüft.
Die Kondition $\kappa$ ist ein Maß dafür, wie stark fehlerbehaftete Eingabewerte
das Ergebnis verfälschen. Je kleiner die Konditionszahl der Matrix, desto geringer
ist der maximale zu erwartene Fehler des Ergebnis.
Die verwendete Matrix hat eine Konditionszahl $\kappa\approx3$ und stellt sich damit
also brauchbare reguläre Matrix vor.
\subsection{Diskussion über die Materialien-Recherce und Wahl der Genauigkeit}
\label{sec:Diskussion2}
Bei "Messing" handelt sich um C27200, welcher im englischsprachigen Raum
als \textit{Common Brass}, gewöhnliches Messing, bezeichnet wird. Die Dichte
wurde für dieses spezielle Messing in der Rechnung benutzt.

Die Absorptionskoeffizienten sind abhängig von der Energie der $\gamma$-Quanten.
Es wurde darauf verzichtet, die Absorptionskoeffizienten für die Materialien durch
Regression für eine exakte Photonenenergie aus den Plots im Anhang zu erhalten.
Der Detektor gibt eine Intensität aus, die sich über einen kleinen Kanalintervall
(Kanal 250-286) erstreckt. Geringfügige Abweichungen der Wellenlänge werden also
als nicht-absorbierte Photonen registriert.
Im Idealfall ist die Quelle monochromatisch und der Detektor nur gegenüber dieser
Frequenz empfindlich. Die so bestimmten Absorptionskoeffizienten berücksichtigen
dabei beide auftretenden Effekte, Photoeffekt und Compton-Streuung, gleichermaßen.

\subsection{Diskussion über die Abweichung von der Anleitung}
\label{sec:Diskussion3}
In der Versuchsanleitung wird versichert, dass die Probenwürfeln aus den Materialien
bestünden, aus denen die homogenen Würfel bestehen.
Unabhängig davon, wie sich die ermittelten Absorptionseigenschaften auf eine Materialhypothese
auswirken, kann der Mittelwert $\bar\mu_i$ herangezogen werden, um ein Material
im Tomographen wiederzuerkennen.

Die Elemente des Probewürfels, die nicht als Leerstellen oder Blei-Würfel erkannt
wurden, haben Absorptionskoeffizienten von  $\mu\approx\SI{0.1}{\per\centi\meter}$.
Dies ist eine wesentliche Abweichung von dem Koeffizienten des Würfel 1-Materials,
$\mu\approx\SI{0.5}{\per\centi\meter}$.
Abweichungen aufgrund von systematischen Fehlern erklären die Abweichung der
Koeffizienten zum Literaturwert. Diese Abweichungen sollten hingegen nicht in den
Messungen zwischen den Würfeln 1 und 4 auftreten, wenn gleichbleibende
Laborbedingungen gesetzt sind. Es wird folglich angenommen, dass ein anderes Material
als Messing vorliegt.
Aus der Erfahrung mit den homogenen Würfeln ist der Koeffzient aus der Messung
kleiner als der Literaturwert. Der nächstgelegene Literaturwert ist der des Delrin.
