\section{Auswertung}
\label{sec:Auswertung}
Die aus dem in Abschnitt \ref{sec:Durchführung} beschriebenen Absorptionskoeffizienten\footnote{kurz: Abs.koeff.} $\mu$
unterscheiden sich von den in der Literatur angegebenen Massenabsorptionskoeffizienten\footnote{kurz:Massenabs.koeff.} $\mu_m$.
Der Zusammenhang der beiden Koeffizienten ist
\begin{equation}
  \mu=\mu_m \cdot \rho,
\end{equation}
wobei $\rho$ die Massenvolumendichte des Materials ist.
Die nach dieser Formel umgerechneten und als Messwerte erwarteten Koeffizienten sind in Tabelle \ref{tab:koeff} aufgetragen.
\begin{table}[b]
  \centering
  \begin{tabular}{l S[table-format=2.2] S[table-format=1.3] S[table-format=1.3]}
    \toprule
    Material& {Dichte}& {Massenabs.koeff.}& {Absorptions-}\\
    &$\si{\gram\per\centi\meter\cubed}$&$\si{\centi\meter\squared\per\gram}$&$\si{\per\centi\meter}$\\
    \midrule
    Delrin&     1.41&  0.082&    0.12\\
    Aluminium&  2.71&  0.075&    0.20\\
    Eisen&      7.70&  0.074&    0.57\\
    Messing&    8.44&  0.074&    0.63\\
    Blei&       11.34& 0.118&    1.34\\
    \bottomrule
  \end{tabular}
  \caption{Absorptions- und Massenabsorptionskoeffizienten in der Umgebung von
  $E_\gamma=\SI{600}{\kilo\electronvolt}$ sowie die Dichten bei Raumtemperatur.
  \cite{dichte_messing}\cite{dichte_pom}\cite{dichte_rest}}
  \label{tab:koeff}
\end{table}
In Tabelle \ref{tab:messwerte_intensitaet} sind die über den ausgesuchten Kanalintervall
integrierten Zerfälle.
In Tabelle \ref{tab:messwerte_intensitaet2} sind die für
die Matrixgleichung \eqref{eq:Matrix} verwendeten Werte des Vektors $\vec{I}$.
Die mithilfe der in Abschnitt \ref{sec:Durchführung} vorgestellten Matrixgleichung
\eqref{eq:Matrix} experimentell bestimmten Absorptionskoeffizienten sind in Tabellen \ref{tab:messwerte_homogen}
und \ref{tab:messwerte_probe} zu finden.

Um die Unsicherheit der Absorptionskoeffizienten zu bestimmten, wird die gemittelte
Unsicherheit in den Intensitäten aus dem Vektor $\vec{I}$ betrachtet.
Mithilfe der Gleichung \eqref{eq:uncert} werden die Unsicherheiten der
Absorptionskoeffizienten bestimmt und in Tabelle \ref{tab:messwerte_homogen}
zusätzlich angegeben.
Die gemittelten Unsicherheiten in der Intensität betragen
\begin{align}
  \sigma_{1}&=0.01,\\
  \sigma_{2}&=0.02,\\
  \sigma_{3}&=0.009.
  \label{eq:uncert_log_intensity}
\end{align}

\begin{table}[p]
  \centering
  \begin{tabular}{llS[table-format=6.0]S[table-format=3.0]}
    \toprule
    {Objekt}&{Projektion}&{Anzahl $N$}&{Unsicherheit $\sqrt{N}$}\\
    \midrule
Leermessung&&            182333&   427\\
Leerwürfel &gerade&  63058&    251\\
 &diagonal&  63098&    251\\
 &halbdiagonal&  62991&    250\\
Würfel 1 & 01&  6361&     79\\
  & 02&  10509&    102\\
  & 03&  10692&    103\\
  & 04&  10940&    104\\
  & 05&  6447&     80\\
  & 06&  11279&    106\\
  & 07&  10401&    101\\
  & 08&  10737&    103\\
  & 09&  10720&    103\\
  & 10&  13163&    114\\
  & 11&  12112&    110\\
  & 12&  11262&    106\\
Würfel 2 & 01&  1105&     33\\
  & 02&  1729&     41\\
  & 03&  2314&     48\\
  & 04&  2539&     50\\
  & 05&  1037&     32\\
  & 06&  4671&     68\\
  & 07&  2223&     47\\
  & 08&  2528&     50\\
  & 09&  3610&     60\\
  & 10&  5777&     76\\
  & 11&  2662&     51\\
  & 12&  3454&     58\\
Würfel 4 & 01&  12503&    111\\
  & 02&  62255&    249\\
  & 03&  3108&     55 \\
  & 04&  61660&    248\\
  & 05&  11767&    108\\
  & 06&  19961&    141\\
  & 07&  19552&    139\\
  & 08&  19589&    139\\
  & 09&  17596&    132\\
  & 10&  15917&    126\\
  & 11&  15361&    123\\
  & 12&  16543&    128\\
  \bottomrule
\end{tabular}
\caption{Die gemessenen Zerfälle pro Messobjekt und Projektion sowie die aus der
Poisson-Verteilung folgende Messunsicherheit.}
\label{tab:messwerte_intensitaet}
\end{table}
\begin{table}
  \centering
  \begin{tabular}{cS[table-format=1.3]@{\,\( \pm \)\,}S[table-format=1.3]S[table-format=1.3]
    @{\,\( \pm \)\,}S[table-format=1.3]S[table-format=1.3]@{\,\( \pm \)\,}S[table-format=1.3]}
    \toprule
    &\multicolumn{3}{c}{Intensitäten}\\
    {Projektion}&\multicolumn{2}{c}{Würfel 1}&\multicolumn{2}{c}{Würfel 2}&\multicolumn{2}{c}{Würfel 4}\\
    \midrule
1&   2.295& 0.013&  4.045&  0.030& 1.619& 0.010\\
2&   1.792& 0.011&  3.597&  0.024& 0.013& 0.006\\
3&   1.775& 0.010&  3.305&  0.021& 3.010& 0.018\\
4&   1.752& 0.010&  3.212&  0.020& 0.022& 0.006\\
5&   2.281& 0.013&  4.108&  0.031& 1.679& 0.010\\
6&   1.721& 0.010&  2.603&  0.015& 1.150& 0.008\\
7&   1.802& 0.011&  3.345&  0.022& 1.171& 0.008\\
8&   1.770& 0.010&  3.217&  0.020& 1.169& 0.008\\
9&   1.771& 0.010&  2.859&  0.017& 1.275& 0.009\\
10&  1.566& 0.010&  2.389&  0.014& 1.376& 0.009\\
11&  1.649& 0.010&  3.164&  0.020& 1.411& 0.009\\
12&  1.722& 0.010&  2.903&  0.017& 1.337& 0.009\\
    \bottomrule
  \end{tabular}
  \caption{Die für die Auswertung benutzte Werte des Vektors $\vec{I}$ und die Unsicherheit.}
  \label{tab:messwerte_intensitaet2}
\end{table}

\begin{table}[ht]
  \centering
  \begin{tabular}{c S[table-format=1.3]@{\,\( \pm \)\,}S[table-format=1.3] S[table-format=-2.2] S[table-format=1.3]@{\,\( \pm \)\,}S[table-format=1.3] S[table-format=-2.2]}
    \toprule
    {Nr.} & \multicolumn{2}{c}{Würfel 1}&{Abweichung zu Lit.} & \multicolumn{2}{c}{Würfel 2}& {Abweichung zu Lit.}\\
    {}&\multicolumn{2}{c}{in $\si{\per\centi\meter}$} &{in \%} &\multicolumn{2}{c}{in $\si{\per\centi\meter}$} &{in \%}\\
    \midrule
     1&   0.619&0.004&   7.36 &  1.342&0.0002&    31.82\\
     2&   0.589&0.002&   2.22 &  1.003&0.0001&    -1.37\\
     3&   0.550&0.004&   -4.6 &  0.975&0.0002&    -4.17\\
     4&   0.547&0.002&   -5.33&  0.918&0.0001&    -9.84\\
     5&   0.522&0.002&   -9.5 &  1.003&0.0001&    -1.44\\
     6&   0.644&0.002&   11.71&  0.992&0.0001&    -2.56\\
     7&   0.580&0.004&   0.64 &  0.998&0.0002&    -1.94\\
     8&   0.636&0.002&   10.35&  1.262&0.0001&    23.99\\
     9&   0.502&0.004&   -12.97& 0.677&0.0002&   -33.53\\
    \bottomrule
  \end{tabular}
  \caption{Aus den gemessenen Intensitäten errechnete Absorptionskoeffizienten der Elementarwürfel,
  homogene Würfel 1 und 2.}
  \label{tab:messwerte_homogen}
\end{table}

Für die homogenen Würfel werden die Durchschnittswerte der Absorptionskoeffizienten% (vgl. Tabelle \ref{tab:messwerte_homogen})
\begin{align}
  \bar \mu_1 &= 0.5765\\
  \bar \mu_2 &= 1.018
  \label{wert:homogen}
\end{align}
ermittelt.
In Tabelle \ref{tab:messwerte_homogen} sind die Abweichungen der Messwerte diesem Mittelwert
 eingetragen, um die Zuverlässigkeit der Messwerte %in Abschnitt \ref{sec:Diskussion}
zu diskutieren. Die Absorptionskoeffizienten in Tabelle \ref{tab:koeff} liegen teilweise nah beieinander.
Die Abweichung der errechneten Koeffizienten in Tabelle \ref{tab:messwerte_homogen} von
zirka $-10\,\%$ bis $10\,\%$ in erster Messung und von bis zu $-33\,\%$ in der zweiten Messung lassen daher vermuten,
dass die Zuordnung der Elementarwürfel zu einem Material bestenfalls grob und unzuverlässig ist.

Der Mittelwert $\bar\mu_2$ aus \eqref{wert:homogen} kann Blei zugeordnet werden.
Der Mittelwert weicht vom Literaturwert um etwa $24\,\%$ nach unten ab.
Obwohl der Mittelwert $\bar\mu_1$ sehr nahe beim Absorptionskoeffizienten
für Eisen liegt, ist es unwahrscheinlich, dass Würfel 1 aus Eisen besteht. Die eingangs
vorgestellten Unsicherheiten und die Abweichungen des Mittelwertes $\bar\mu_2$ legen in diesem Fall einen Hypothesen-Fehler nahe.
Bestünde der erste Würfel aus Messing, so betrüge die Abweichungen von $\bar\mu_1$
etwa $8,5\,\%$ nach unten.
Diese Abweichung erscheint plausibel im Anbetracht der anderen auftretenden Abweichungen.
Der Würfel 1 besteht demnach vermutlich aus Blei, der Würfel 2 aus Messing.
\begin{table}[ht]
  \centering
  \begin{tabular}{cS[table-format=1.3]@{\,\( \pm \)\,}S[table-format=1.3]S[table-format=1.0]S[table-format=2.1]}
    \toprule
    \multicolumn{3}{c}{Proben-Würfel Nr. 4}&&{Abweichung zu Lit.}\\
    {}&\multicolumn{2}{c}{in $\si{\per\centi\meter}$} &{Art}&{in \%}\\
    \midrule
     1&   0.04&   0.00003& 1&    -66.6\\
     2&  -0.009&  0.00002& 0&    -0.9\\
     3&   0.04&   0.00003& 1&    -66.6\\
     4&   1.01&   0.00002& 2&    -24.6\\
     5&   1.092&  0.00002& 2&    -18.5\\
     6&   0.949&  0.00002& 2&    -29.2\\
     7&   0.028&  0.00003& 1&    -76.6\\
     8&   -0.018& 0.00002& 0&    -1.8\\
     9&   0.07&   0.00003& 1&    -41.6\\
    \bottomrule
  \end{tabular}
  \caption{Aus den gemessenen Intensitäten errechnete Absorptionskoeffizienten der Elementarwürfel,
  Proben-Würfel Nr.\,4.}
  \label{tab:messwerte_probe}
\end{table}
Die negativen Koeffizienten, die bei den homogenen Würfeln nicht
aufgetreten sind, können als Leerstellen interpretiert werden.
In dieser Auswertung wird das Abklingen der $\gamma$-Strahlung in Luft vernachlässigt.
Es kann angenommen werden, dass nur wenig Leerstellen auftreten; ein Würfelaufbau
mit vielen Leerstellen ist unbeständig und würde Anordnungsfehler verursachen,
die die Auswertung sehr schwierig gestalten.
In Tabelle \ref{tab:messwerte_probe} ist in eigener Spalte die Art beschrieben,
zu welcher der Würfel geordnet werden kann. Art 0 ist eine Leerstelle, Art 1 ist Delrin,
Art 2 ist Blei.

Es ist überzeugend, die Würfel 4,5,6 als Bleiwürfel zu identifizieren respektive sie
dem Material von Würfel 2 zuzuordnen.
Die Abweichungen der Elementarwürfel von dem Literaturwert ist in gleicher Größenordung
wie bei der Messung an Würfel 2.
Die übrigen Würfel haben auffallend geringe Koeffizienten und damit vermutlich aus Delrin.
Da Messing im ersten Teil der Auswertung mit Koeffizienten $\mu\approx\SI{0.5}{\per\centi\meter}$
erkannt wurde, ist Messing nur sehr unwahrscheinlich das Material der Elementarwürfel.
Da Delrin als Reinmaterial nicht untersucht wurde, kann keine verlässliche Aussage über die
hohen Abweichungen getroffen werden.
Ausführliche Diskussion ist hierzu in Abschnitt \ref{sec:Diskussion}.
