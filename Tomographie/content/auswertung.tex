\section{Auswertung}
\label{sec:Auswertung}
Die aus dem in Abschnitt \ref{sec:Durchführung} beschriebenen Absorptionskoeffizienten\footnote{kurz: Abs.koeff.} $\mu$
unterscheiden sich von den in der Literatur angegebenen Massenabsorptionskoeffizienten\footnote{kurz:Massenabs.koeff.} $\mu_m$.
Der Zusammenhang der beiden Koeffizienten ist
\begin{equation}
  \mu=\mu_m \cdot \rho,
\end{equation}
wobei $\rho$ die Massenvolumendichte des Materials ist.
Die umgerechneten und erwarteten Koeffizienten sind in Tabelle \ref{tab:koeff} aufgetragen.
\begin{table}[t]
  \centering
  \begin{tabular}{l S[table-format=2.2] S[table-format=1.3] S[table-format=1.3]}
    \toprule
    Material& {Dichte}& {Massenabs.koeff.}& {Absorptions-}\\
    &$\si{\gram\per\centi\meter\cubed}$&$\si{\centi\meter\squared\per\gram}$&$\si{\per\centi\meter}$\\
    \midrule
    Delrin&     1.41&  0.082&    0.12\\
    Aluminium&  2.71&  0.075&    0.20\\
    Eisen&      7.70&  0.074&    0.57\\
    Messing&    8.44&  0.074&    0.63\\
    Blei&       11.34& 0.118&    1.34\\
    \bottomrule
  \end{tabular}
  \caption{Absorptions- und Massenabsorptionskoeffizienten in der Umgebung von
  $E_\gamma=\SI{600}{\kilo\electronvolt}$ sowie die Dichten bei Raumtemperatur.
  \cite{dichte_messing}\cite{dichte_pom}\cite{dichte_rest}}
  \label{tab:koeff}
\end{table}
Die mithilfe des in Abschnitt \ref{sec:Durchführung} vorgestellten Verfahrens experimentell
bestimmten Absorptionskoeffizienten sind in Tabellen \ref{tab:messwerte_homogen}
und \ref{tab:messwerte_probe} zu finden.
\begin{table}[ht]
  \centering
  \begin{tabular}{c S[table-format=1.3] S[table-format=-2.2] S[table-format=1.2] S[table-format=-2.2]}
    \toprule
    {Nr.} & {Würfel 1}&{Abweichung} & {Würfel 2}& {Abweichung}\\
    {}&{in $\si{\centi\meter\squared\per\gram}$} &{in \%} &{in $\si{\centi\meter\squared\per\gram}$} &{in \%}\\
    \midrule
     1&   0.6189&   7.36 &  1.3419&    31.82\\
     2&   0.5893&   2.22 &  1.0039&    -1.37\\
     3&   0.5499&   -4.6 &  0.9754&    -4.17\\
     4&   0.5472&   -5.33&  0.9178&    -9.84\\
     5&   0.5217&   -9.5 &  1.0033&    -1.44\\
     6&   0.6440&   11.71&  0.9919&    -2.56\\
     7&   0.5802&   0.64 &  0.9982&    -1.94\\
     8&   0.6361&   10.35&  1.2622&    23.99\\
     9&   0.5016&   -12.97& 0.6766&    -33.53\\
    \bottomrule
  \end{tabular}
  \caption{Aus den gemessenen Intensitäten errechnete Absorptionskoeffizienten der Elementarwürfel,
  homogene Würfel 1 und 2.}
  \label{tab:messwerte_homogen}
\end{table}

Für die homogenen Würfel werden die Mittelwerte $\bar \mu$
\begin{align}
  \bar \mu_1 &= 0.5765\\
  \bar \mu_2 &= 1.018
  \label{wert:homogen}
\end{align}
ermittelt (im wahren Sinne :D ).
In Tabelle \ref{tab:messwerte_homogen} sind die Abweichungen der Messwerte vom
Mittelwert eingetragen, um die Zuverlässigkeit der Messwerte %in Abschnitt \ref{sec:Diskussion}
zu diskutieren. Die Absorptionskoeffizienten in Tabelle \ref{tab:koeff} liegen teilweise nah beieinander.
Die Abweichung der errechneten Koeffizienten in Tabelle \ref{tab:messwerte_homogen} von
zirka $-10\,\%$ bis $10\,\%$ in erster Messung und von bis zu $-33\,\%$ in der zweiten Messung lassen daher vermuten,
dass die Zuordnung der Elementarwürfel zu einem Material bestenfalls grob und unzuverlässig ist.

Der Mittelwert $\bar\mu_2$ aus \eqref{wert:homogen} kann Blei zugeordnet werden.
Der Mittelwert weicht vom Literaturwert um etwa $24\,\%$ nach unten ab.
Obwohl der Mittelwert $\bar\mu_1$ sehr nahe beim Absorptionskoeffizienten
für Eisen liegt, ist es unwahrscheinlich, dass Würfel 1 aus Eisen besteht. Die eingangs
vorgestellten Unsicherheiten und die Abweichungen des Mittelwertes $\bar\mu_2$ legen in diesem Fall einen Hypothesen-Fehler nahe.
Bestünde der erste Würfel aus Messing, so betrüge die Abweichungen von $\bar\mu_1$
etwa $8,5\,\%$ nach unten.
Diese Abweichung erscheint plausibel im Anbetracht der anderen auftretenden Abweichungen.
Der Würfel 1 besteht demnach vermutlich aus Blei, der Würfel 2 aus Messing.
\begin{table}[ht]
  \centering
  \begin{tabular}{cS[table-format=1.3]S[table-format=1.0]S[table-format=2.1]}
    \toprule
    \multicolumn{3}{c}{Proben-Würfel Nr. 4}&{Abweichung zu $\mu_m$}\\
    {}&{in $\si{\per\centi\meter}$} &{Art}&{in \%}\\
    \midrule
    Würfel 1&   0.04&   1&    -66.6\\
    Würfel 2&  -0.009&  0&    -0.9\\
    Würfel 3&   0.04&   1&    -66.6\\
    Würfel 4&   1.01&   2&    -24.6\\
    Würfel 5&   1.092&  2&    -18.5\\
    Würfel 6&   0.949&  2&    -29.2\\
    Würfel 7&   0.028&  1&    -76.6\\
    Würfel 8&   -0.018& 0&    -1.8\\
    Würfel 9&   0.07&   1&    -41.6\\
    \bottomrule
  \end{tabular}
  \caption{Aus den gemessenen Intensitäten errechnete Absorptionskoeffizienten der Elementarwürfel,
  Proben-Würfel Nr.\,4.}
  \label{tab:messwerte_probe}
\end{table}
Die negativen Koeffizienten, die bei den homogenen Würfeln nicht
aufgetreten sind, können als Leerstellen interpretiert werden.
In dieser Auswertung wird das Abklingen der $\gamma$-Strahlung in Luft vernachlässigt.
Es kann angenommen werden, dass nur wenig Leerstellen auftreten; ein Würfelaufbau
mit vielen Leerstellen ist unbeständig und würde Anordnungsfehler verursachen,
die die Auswertung sehr schwierig gestalten.
In Tabelle \ref{tab:messwerte_probe} ist in eigener Spalte die Art beschrieben,
zu welcher der Würfel geordnet werden kann. Art 0 ist eine Leerstelle, Art 1 ist Delrin,
Art 2 ist Blei.

Es ist überzeugend, die Würfel 4,5,6 als Bleiwürfel zu identifizieren respektive sie
dem Material von Würfel 2 zuzuordnen.
Die Abweichungen der Elementarwürfel von dem Literaturwert ist in gleicher Größenordung
wie bei der Messung an Würfel 2.
Die übrigen Würfel haben auffallend geringe Koeffizienten und damit vermutlich aus Delrin.
Da Messing im ersten Teil der Auswertung mit Koeffizienten $\mu\approx\SI{0.5}{\per\centi\meter}$
erkannt wurde, ist Messing nur sehr unwahrscheinlich das Material der Elementarwürfel.
Da Delrin als Reinmaterial nicht untersucht wurde, kann keine verlässliche Aussage über die
hohen Abweichungen getroffen werden.
Ausführliche Diskussion ist hierzu in Abschnitt \ref{sec:Diskussion}.
