\section{Diskussion}
\label{sec:Diskussion}
Bei dem Versuch treten, durch den sehr empfindlichen Aufbau, viele Fehlerquellen auf. Durch die sehr kleinen Magnetfelder und kleinen Ströme, gibt es sehr schnell starke Abweichungen durch Einflüsse von der Umgebung. So können umliegende Experimente mit starken Magnetfeldern für Veränderungen sorgen.
Die Stellschrauben zum Abgleich der Brücke sind sehr locker und empfindlich, sodass ein Wackeln am Tisch oder auch auf dem Boden diese verstellen und damit die Form der Resonanzkurve verändern. 
Trotz dieser möglichen Fehlerquellen ist es gelungen, sehr gute Ergebnisse zu erzielen.
Der Wert des gyromagnetischen Verhältnis und des Erdmagnetfelds stimmen mit den Literaturwerten fast exakt überein.
Dies ist einer sehr sorgfältigen Messung und Auswertung zu verdanken.
