\section{Zielsetzung}
\nocite{skript}
Zwischen dem Bahndrehimpuls $\vec{L}$ und dem magnetischen Moment $\vec{m}$
besteht in der klassischen Physik der Zusammenhang  $\vec{m} =\gamma \vec{L}$ mit
dem gyromagnetischen Verhältnis $\gamma$.
Jedoch können für gebundene Elektronen eines Atoms mit verschwindenem Bahndrehimpuls
oder für freie Elektronen eine magnetisches Moment gemessen werden.
Ziel des Experimentes ist die Vermessung und skizzenhafte Herleitung des magnetischen
Momentes eines freien Elektrons.

\section{Theorie}
\label{sec:Theorie}
\subsection{Quantenmechanische Beschreibung des magnetischen Moments von einem gebundenen Elektron}
Für ein Atom in der Einelektronen-Näherung ist die Wellenfunktion
\begin{equation}
  \Psi_\text{n,l,m}(r,\theta,\phi) = R_\text{n,l}(r)\cdot \Theta_\text{l,m}(\theta)\cdot \Phi(\phi) \
  = R_\text{n,l}(r)\cdot \Theta_\text{l,m}(\theta)\cdot \frac{\exp\left(im\phi\right)}{\sqrt{2\pi}}
  \label{eq:1e_approx}
\end{equation}
mit den Quantenzahlen: Hauptquantenzahl $n$, Nebenquantenzahl $l$ und
magnetische Orientierung $m$.
Um die Normierung der Wellenfunktion $\Psi_\text{n,l,m}(r,\theta,\phi)$ zu aufrecht zu halten,
sind alle Faktorfunktionen über das Volumenintegral normiert.
Die quantenmechanische Stromdichte
\begin{align}
  \vec{S}&=\frac{\hbar}{2im_0}(\Psi^{*}\nabla\Psi - \Psi\nabla\Psi^{*})\\
          &=\frac{\hbar}{m_0}\frac{R^2\Theta^2}{2\pi}\frac{m}{r\sin(\theta)}
        \label{eq:stromdichte}
\end{align}
ergibt nach Mulitplikation mit der Ladung die Stromdichte
\begin{equation}
  j_\text{el}= - e_0 \vec{S}.
  \label{eq:stromdichte2}
\end{equation}
Zu beachten ist, dass in der Wellenfunktion \eqref{eq:1e_approx} nur der $\Phi$-Anteil
komplex ist und dass dadurch die radialen und $\Theta$-Anteile in \eqref{eq:stromdichte}
verschwinden.
Mit \eqref{eq:stromdichte} wird der Strom entlang des Azimuth-Winkels
\begin{align}
  \mathup{d}I_\phi &= j_\text{el} \mathup{d}A_\phi\\
          &= j_\text{el} r \mathup{d}\theta \mathup{d}r
\end{align}
und das magnetische Moment als Produkt dieses Stromes und der umströmten Fläche
\begin{align}
  \mathup{d}\mu&=\mathup{d}I_\phi \cdot A\\
            &=j_\text{el} r \mathup{d}\theta \mathup{d}r \cdot \pi r^2\sin^2(\theta)
\end{align}
gefunden.
Nach Ausführen der trivialen Integrationen - die Normierung der Funktionen $R$ und $\Theta$ ist maßgeblich-
wird das magnetische Moment
\begin{align}
  \mu&=-\hbar m \frac{e_0}{2m_0}\int\limits_0^\infty r^2 R^2(r)\mathup{d}r \
  \int\limits_0^{2\pi} \Theta^2(\theta)\sin(\theta)\mathup{d}\theta\\
      &=-\hbar m \frac{e_0}{2m_0}\\
      &=: m \mu_B
\end{align}
gefunden.
Die Abkürzung $\mu_B$ besteht nur aus Naturkonstanten und wird Bohrsches
Magneton bezeichnet, es verknüpft die Magnetquantenzahl und das magnetische Moment;
\begin{equation}
  \mu_B = \SI{9,274015(3)e-24}{\joule\per\tesla}.
  \label{eq:bohrmag}
\end{equation}
Es folgt also, dass das magnetische Moment quantisiert ist.
\subsection{Richtungsquantisierung des \texorpdfstring{magnetischen Momentes $\mu$}{magnetischen Momentes}
und die Aufspaltung der Energieniveaus im Magnetfeld}
Die Komponente des Bahndrehimpulses $\vec{L}$ entlang einer ausgezeichneten Richtung
ist abhängig von der magnetischen Orientierungsquantenzahl $m$.
Wird die Richtung des Magnetfeldes willkürlich in $z$-Richtung festgelegt,
kann die $z$-Komponente des Bahndrehimpulses $L_z$ als
\begin{equation}
  L_z = m\cdot\hbar
\end{equation}
festgelegt werden.
Die Quantenzahl $m$ ist ganzzahlig und durch die Forderung $L_z \le |\vec{L}|$
beschränkt, $m \in [0,\pm 1,\pm 2,..., \pm l]$ und $\vert m \vert = 2l+1$.
Diese Quantisierung ist in den Energieniveaus sichtbar. Die Energie eines magnetischen
Dipols $\vec{M}=L_z\vec{e_z}$ in einem äußeren Magnetfeld $\vec{B}$ ist
\begin{equation}
  E = -\vec{M}\cdot\vec{B} = -m\hbar\cdot B.
\end{equation}
Das Aufspalten der Energieniveaus bei Anwesenheit eines Magnetfeldes ist der
Zeeman-Effekt.
Er kann dadurch sichtbar gemacht werden, dass die Kraft auf einen magnetischen
Dipol im inhomogenen Feld,
\begin{equation}
  F= -\nabla E,
\end{equation}
im Falle vom oben beschriebenen gebundenen Elektron ebenfalls quantisiert ist.
Ein Detektor könnte die Ablenkung des Elektronenstrahls anhand der $\vert m \vert = 2l+1$
verschiedenen Richtungen anzeigen.

\subsection{Auswirkung des Elektronenspins auf die Energieniveaus}
Durch den Stern-Gerlach-Versuch wurde gezeigt, dass auch Teilchenstrahlen abgelenkt
werden, wenn die enthaltenen gebundenen Elektronen keinen Bahndrehimpuls haben.
Im Originalaufbau werden Silberatome benutzt, deren äußeres Elektron keinen
Bahndrehimpuls hat.
Ein Detektor zeigt an, dass der Silberatom-Strahl betragsmäßig gleich in zwei
Richtungen abgelenkt werden.
Aus der Anzahl der erlaubten Werte für $m$ folgt, dass ein Bahndrehimpuls-artiger
Zustand $\vec{S}$ mit $l_\text{Spin} =\sfrac{1}{2}$ neben dem konventionellen
Bahndrehimpuls $\vec{L}$ existiert.
Die erlaubten Quantenzahlen der magnetischen Orientierung sind halbzahlig,
$\m_\text{Spin}=\pm\sfrac{1}{2}$.
Aus dieser Analogie zum Bahndrehimpuls wird der zusätzliche Zustand \emph{Spin}
oder \emph{Eigendrehimpuls} genannt, obwohl Elektronen als
punktförmige Teilchen nicht rotieren können.
Das magnetische Moment eines Elektronenspin ist
\begin{equation}
  \mu_\text{Spin} = \pm g m_\text{Spin} \mu_B
\end{equation}
mit dem eingeführten Landé-Faktor oder gyromagnetischen Verhältnis $g \approx 2$.
