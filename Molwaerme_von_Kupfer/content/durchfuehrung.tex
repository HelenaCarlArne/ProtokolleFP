\section{Aufbau}
\begin{figure}
	\centering
	\includegraphics[width=0.6\textwidth,]{graphics/aufbau.png}
	\caption{Skizzenhafter Aufbau des Versuchs. \cite{skript}}
	\label{fig:aufbau}
\end{figure}
Der Versuchsaufbau ist in \ref{fig:aufbau} dargestellt.
In einem \emph{Dewargefäß} befindet sich der \emph{Rezipient} und im Rezipienten befindet sich die \emph{Probe mit Heizwicklung}, das von einem \emph{Kupfer-Zylinder mit Heizwicklung} umgeben ist.
Die Heizwicklungen der Probe und des Kupferzylinders arbeiten unabhängig voneinander.
Der Rezipient ist mit Gas befüllbar und kann mittels einer Vakuumpumpe evakuiert werden. 
Die Temperaturmessung von Probe und Kupferzylinder wird durch zwei \emph{Pt-100-Messwiderständen}, je eine an beiden Heizungen, ermöglicht.


\section{Durchführung}
Es wird die Temperatur-Abhängigkeit der Molwärme $C_p$ von Kupfer im
Bereich von $\SI{80}{\kelvin}$ bis $\SI{300}{\kelvin}$ gemessen.
Zum Kühlen der Probe wird der Rezipient mit gasförmigen Helium geflutet
und das Dewar-Gefäß mit flüssigem Stickstoff gefüllt.
Auf diese Weise stellt sich ein Temperaturgleichgewicht zwischen Probe und Stickstoff ein und die Probe wird auf die Temperatur flüssigen Stickstoffs $T=\SI{90}{\kelvin}$ \cite{stickstoff} gekühlt.
Erreichen sowohl Gehäuse als auch die Probe die Temperatur des Stickstoffs, 
wird der Rezipient evakuiert und der Druck bis zum Ende der Messungen mittels der Vakuumpumpe gering gehalten, um Wärmestrom durch verbliebenes Gas und Konvektionsstrom als Fehlerquelle gering zu halten.
Über die Probenheizung kann der Probe Wärme zugeführt werden, 
während die davon unabhängige Heizung die Gehäusetemperatur auf Probentemperatur hält.
Es wird die Probenheizspannung $U$, Probenheizstrom $I$ und die Zeit $t$ in Abschnitten notiert, 
in welchen die Probe um $\SIrange{7}{11}{\kelvin}$ erwärmt wird.