\section{Zielsetzung}
Es wird die Molwärme, 
das heißt die aufgenommene Energie bei Temperatur-Erhöhung, 
von einer Kupferprobe experimentell bestimmt. 
Hierzu werden drei Theorien -- das klassische Modell, das Einstein-Modell und das Debye-Modell -- zunächst dargelegt und anschließend die Molwärme einer Kupferprobe bestimmt.
Die Messergebnisse werden mit den Vorhersagen des Debye-Modells verglichen.

\section{Theorie}
\label{sec:Theorie}
Die Molwärme $C$ ist eine thermodynamische Größe, die die Änderung der inneren Energie $U$ unter Änderung der Temperatur $T$ beschreibt.
Hierzu wird zwischen den Molwärmen bei konstantem Druck
\begin{align}
	C_p = \left(\frac{\partial U}{\partial T}\right)_p
	\label{eq:molwaermep}
	\intertext{und bei konstantem Volumen}
	C_V = \left(\frac{\partial U}{\partial T}\right)_V
	\label{eq:molwaermev}
\end{align}
unterschieden.
Der Zusammenhang zwischen diesen Größen wird durch
\begin{equation}
	C_p-C_V = 9\alpha^2\kappa V_0 T
	\label{eq:molwaermeconv}
\end{equation}
beschrieben,
wobei $\alpha$ der lineare Ausdehnungskoeffizient, $\kappa$ der Kompressionsmodul und $V_0$ das Molvolumen sind.

\subsection{Die klassische Theorie der Molwärme}
In einem kristallinen Festkörper sind die Atome durch Gitterkräfte an ihre Ruhelage gebunden. 
Sie können daher nur Schwingungen in den drei senkrecht aufeinander stehenden Raumrichtungen ausführen. 
Die kinetische Energie ist dabei nach dem Äquipartitionstheorem gleichmäßig auf die Bewegungsfreiheitsgrade verteilt. 
Im zeitlichen Mittel besitzen die Atome 
pro Freiheitsgrad
eine Energie von $\frac{1}{2}k_\text{B}T$. 
Bei eine harmonischen Schwingung sind kinetische und potentielle Energie im Mittel gleich, 
daher wird nach klassischer Betrachtung eine mittlere Energie von
\begin{equation*}
	\langle u \rangle = 2\cdot3\cdot\frac{1}{2}k_\text B T
\end{equation*}
angenommen.
Im Kristall ergibt sich mit der Avogadro-Konstanten $N_\text A$ bei der Teilchenzahl $N = N_\text A = \SI{1}{mol}$ für die Energie im Allgemeinen
\begin{equation}
	U = N_\text A \langle u \rangle
	\label{eq:gesamtenergie}
\end{equation}
und im Weiteren für die klassische Theorie
\begin{equation}
	U_\text{klass} = 3k_\text B N_\text A T = 3RT.
\end{equation}
Damit kann die spezifische Molwärme bei konstantem Volumen mit
\begin{equation}
	C_V = \left(\frac{\partial U}{\partial T}\right)_V = 3R
\end{equation}
berechnet werden.

Die klassische Theorie steht im Widerspruch zu den experimentellen Ergebnissen.
Es ist bekannt, 
dass die Molwärme für niedrige Temperaturen 
von dem Material und von der momentanen Temperatur $T$ abhängig ist.


\subsection{Das Einstein-Modell}
Die klassischen Theorie missachtet die Quantelung der Energien, insbesondere die Energiequantelung beim haromischen Oszillator. 
Im Einstein-Modell wird angenommen, 
dass die Atome mit der Frequenz $\omega$ schwingen und  nur Energien, die ein Vielfaches von $\hbar\omega$ sind, annehmen oder abgeben können.
Zur Berechnung der mittleren Energie $\langle u \rangle _\text{Einstein}$ wird im Kontrast zum Äquipartitionstheorem die Verteilung $W(n)$ der Energien betrachtet. 
$W(n)$ beschreibt die Wahrscheinlichkeit, 
dass ein harmonischer Oszillator 
bei der Temperatur $T$ im thermischen Gleichgewicht die Energie $n\hbar\omega$ besitzt.
Mit der Boltzmann-Verteilung wird
\begin{equation*}
	W(n) = \exp{\left(-\frac{n\hbar \omega}{k_\text B T}\right)}
\end{equation*}
angegeben, womit 
\begin{equation*}
	\langle u \rangle _\text{Einstein} = \frac{\sum\limits_n W(n)\cdot E(n)}{\sum\limits_n W(n)}=	\frac{\hbar\omega}{\exp{\left(\frac{\hbar\omega}{k_\text B T}\right)} - 1}  < k_\text B T
\end{equation*}
für die mittlere Energie folgt.
Durch diese Formel kann durch \eqref{eq:molwaermev} auf die Molwärme
\begin{equation}
	C_{V_\text{Einstein}} = \frac{d}{dT}3 N_\text A \frac{\hbar\omega}{\exp{\left(\frac{\hbar\omega}{k_\text B T}\right)} - 1} = 3R \frac{\hbar^2\omega^2}{k_\text B ^2 T^2} \frac{\exp{\left(\frac{\hbar\omega}{k_\text B T}\right)}}{\left(\exp{\left(\frac{\hbar\omega}{k_\text B T}\right)}-1\right)^2}\,\text.
	\label{eqn:cv}
\end{equation}
geschlossen werden.

Im Einklang mit dem klassischen Modell und der experimentellen Erfahrung gilt für große Temperaturen der Grenzwert
\begin{equation}
	\lim_{T\to\infty} C_{V_\text{Einstein}} = 3R.
\end{equation}
Dies besagt das Dulong--Petit-Gesetz.
Die monofrequente Näherung stellt dennoch eine grobe Näherung dar, 
sodass bei tiefen Temperaturen große Abweichungen auftreten.

\subsection{Das Debye-Modell}
\label{sec:theorie_debye}
Zusätzlich zu den Annahmen im Einstein-Modell, wird die spektrale Verteilung $Z(\omega)$ der Schwingungsfrequenzen $\omega$ berücksichtigt.
Es wird für die Berechnung der Energie $U$ über alle Frequenzen integriert, sodass die Molwärme zu
\begin{equation}
	C_V = \frac{\mathup d}{\mathup d T}\int_0^{\omega_\text{max}} Z(\omega) \langle u \rangle _\text{Einstein} \mathup d \omega= \frac{\mathup d}{\mathup d T}\int_0^{\omega_\text{max}} Z(\omega) \frac{\hbar\omega}{\exp{\left(\frac{\hbar\omega}{k_\text B T}\right)} - 1} d\omega
	\label{eqn:cvn}
\end{equation}
bestimmt ist.
Die Verteilung $Z(\omega)$ kann eine sehr komplizierte Gestalt haben. 
Wenn der Kristall weder stark dispersiv noch anisotrop ist, 
kann in guter Näherung angenommen werden, 
dass die Eigenfrequenzen in einem Würfel der Länge $L$ abgezählt werden können.
%Es wird also klassische Physik mit Quantenmechanik verknüpft.
Die Spektraldichte beträgt dann mit der allgemeine Schallgeschwindigkeit $v$ 
\begin{equation*}
	Z(\omega)d\omega = \frac{3L^3}{2\pi^2v^3}\omega^2 d\omega
\end{equation*}
oder, wenn zwischen longitudinaler Schallgeschwindigkeit $v_\text l$ und transversaler Schallgeschwindigkeit $v_\text t$ unterschieden wird,
\begin{equation}
	Z(\omega)d\omega = \frac{L^3}{2\pi^2}\omega^2\left(\frac{1}{v_\text{l}^3}+\frac{2}{v_{\text{t}}^3}\right)d\omega.
	\label{eqn:z}
\end{equation}

Da ein endlicher Kristall endlich viele Eigenschwingungen besitzt, kann eine obere Grenzfrequenz $\omega_\text D$ berechnet werden. 
Unter der Annahme, der Kristall bestehe aus $N_\text L$ Atomen, heißt die Grenzfrequenz $\omega_\text{D}$ mit 
\begin{equation}
	\int_0^{\omega_\text D} Z(\omega) d\omega = 3 N_\text L
	\label{eqn:wd}
\end{equation}
Debye-Frequenz. 
Dabei ist $N_\text L$ die Avogadro-Konstante.
Damit ist 
\begin{equation}
	Z(\omega)d\omega = \frac{9 N_\text L}{\omega_\text D ^3}\omega^2 d\omega
	\label{eqn:z2}
\end{equation}
die Verteilungsfunktion.
Es ergibt sich für die Molwärme
\begin{equation}
	C_{V_\text{Debye}} = \frac{d}{dT}\left[9 N_\text L k_\text B T \left(\frac{T}{\theta_\text D}\right)^3 \int_0^{\frac{\theta_\text D}{T}} \frac{x^3}{\mathup e^x -1} dx \right] = 9 R \left(\frac{T}{\theta_\text D}\right)^3 \int_0^{\frac{\theta_\text D}{T}} \frac{x^4\mathup e^x}{\mathup (e^x -1)^2} dx
	\label{eqn:cvd}
\end{equation}
mit den Abkürzungen
\begin{equation*}
	x := \frac{\hbar\omega}{k_\text B T}\qquad\text{und}\qquad \frac{\theta_\text D}{T} :=\frac{\hbar\omega_\text D }{k_\text B T}
\end{equation*}
in der Debye-Näherung.
Im Einklang mit den vorangegangen Modellen und der experimentellen Erfahrung gilt für große Temperaturen der Grenzwert 
\begin{equation}
	\lim_{T\to\infty} C_{V_\text{Einstein}} = 3R.
\end{equation}
Zusätzlich wird erfasst, dass die Molwärme bei tiefen Temperaturen  proportional zu $T^3$ ist. 
Damit ist das Debye-Modell für das Tieftemperaturverhalten eine bessere Näherung als das Einstein-Modell.