\section{Auswertung}
\label{sec:Auswertung}
Die Funktionsweise des Versuchsaufbaus mit dem Vorgehen nach Abschnitt \ref{sec:durchfuehrung_1} wurde getestet
und die Messung gestartet, 
nachdem das System ausreichend aufgeheizt ist.

Die Amplitude $A$ ist ein Maß für die Leistung $P$ und wird in willkürlichen Einheiten oder Skalenteile gemessen,
da Berechnung nur bei Leistungsverhältnisse stattfinden.
%
\subsection{Untersuchung der Moden des Reflexklystrons}
%
\begin{table}
\centering
\caption{Messwerte zur Untersuchung der Moden des Reflexklystrons.}
\begin{tabular}{c cccccc}
	\toprule
	{\# Mode} &$\frac{U_-}{\si{\volt}}$ &$\frac{U_\text{max}}{\si{\volt}}$ &$\frac{U_+}{\si{\volt}}$ &$A_\text{Messung}$ &$A_\text{Fit}$ &$\frac{f_0}{\si{\giga\hertz}}$\\
	\midrule
		1 &200 	&220 &230 	&6.3 &7.087 &9.000 \\
		2 &120 	&140 &150 	&6.3 &7.087 &9.003 \\
		3 &70 	& 80 &90 	&5.3 &5.3 &9.010 \\
	\bottomrule
\end{tabular}
\label{tab:moden}
\end{table}

Die Moden des Reflexklystrons können bestimmt werden,
indem die Reflektorspannungen $U_-$, $U_\text{max}$, $U_+$ und die Amplitude $A$ aufgenommen werden, 
bei welchen ein Oszillogram nach Abbildung \ref{fig:moden} angezeigt wird.
Die Frequenz $f_0$ wird bestimmt, indem der Frequenzmesser so eingestellt wird, sodass exakt die Spitze der Mode in Abbildung \ref{fig:moden} geringfügig eindellt.
Die Messwerte sind in Tabelle \ref{tab:moden} dargestellt und werden in Abbildung \ref{plt:moden} als Messpunkte und Fit einer quadratischen Funktion 
%
\begin{equation}
	P(U)= a\cdot x^2 + b\cdot x + c
	\label{eq:quadfkt}
\end{equation}
%
aufgetragen.
In dieser Abbildung sind die Messpunkte bei den beiden größten Moden nicht an dem Hochpunkt der Modenkurve, 
da die Spannungen $U_\pm$ nicht symmetrisch um das Maximum liegen.
Dies ist auf zu geringe Ablesbarkeit des Voltmeters für die Reflektorspannung zurückzuführen und stellt eine Fehlerquelle dar.
\begin{figure}
  \centering
  \includegraphics[width=\textwidth]{pc/plot.pdf}
  \caption{Darstellung der gemessenen Moden des Reflexklystrons.}
  \label{plt:moden}
\end{figure}

%
\begin{table}
\centering
\caption{Messwerte für die elektronische Abstimmung.}
\begin{tabular}{c cc}
	\toprule
	&{untere Grenze} &{obere Grenze}\\
	\midrule
		{Spannung/$\si{\volt}$} &210 &230 \\
		{Frequenz/$\si{\mega\hertz}$} &8978 &9023\\
	\bottomrule
\end{tabular}
\label{tab:elktr_abstimmung}
\end{table}
%
Die elektronische Abstimmung benötigt, gemessen an der größtmöglichen Mode, die Spannungen $U_{-\sfrac{1}{2}}$, $U_{\sfrac{1}{2}}$ und die dazugehörigen Frequenzen $f_{-\sfrac{1}{2}}$, $f_{\sfrac{1}{2}}$ bei halber Leistung.
Die Abstimmung ist durch
\begin{equation}
	A=\frac{f_{\sfrac{1}{2}}-f_{-\sfrac{1}{2}}}{U_{\sfrac{1}{2}}-U_{-\sfrac{1}{2}}} = \SI{2.25}{\mega\hertz\per\volt}
\end{equation}
gegeben;
Messwerte sind in Tabelle \ref{tab:elktr_abstimmung} aufgezählt.

\subsection{Untersuchung von Frequenz, Wellenlänge und Dämpfung}
Zur Berechnung der Wellenlänge 
\begin{equation}
	\lambda_\text c = 2a
\end{equation} 
werden die Abmessungen des Hohlleiters, im Detail die längere Innenseite $a$ des Leiterquerschnitts (vgl. Abb. \ref{fig:hohlleitung}), benötigt. 
Mit der Hohlleiter-Wellenlänge $\lambda_\text g$ als zweifachen Abstand zweier Intensitätsminima wird die Frequenz 
\begin{equation}
	f=c\sqrt{\frac{1}{\lambda^2_\text g}+\frac{1}{4a^2}}
\end{equation}
bestimmt und als Grundlage für die Dämpfungsmessung benützt.
Die Abmessungen, die Wellenlänge $\lambda_\text g$ und Frequenz $f$ sind in Tabelle \ref{tab:flambdagamma}
gezeigt. 
\begin{table}
\centering
\caption{Messwerte zur Untersuchung der Moden des Reflexklystrons.}
\begin{tabular}{ccccc}
	\toprule
	$\frac{f_\text{Messung}}{\si{\mega\hertz}}$& \
	$\frac{\mathup{\Delta}x_\text{Minima}}{\si{\milli\meter}}$& \
	$\frac{a}{\si{\milli\meter}}$& \
	$\frac{\lambda_\text g}{\si{\milli\meter}}$& \
	$\frac{f_\text{Berechnung}}{\si{\mega\hertz}}$\\
	\midrule
		9023&	24.1&	22.62&	0&	0\\
	\bottomrule
\end{tabular}
\label{tab:flambdagamma}
\end{table}