\section{Diskussion}
\label{sec:Diskussion}
Da durchgehend nur wenig Werte aufgenommen werden, kann eine ausführliche Fehlerdiskussion und die Berechnung von Standardabweichungen der Stichproben nicht erfolgen.
Es kann keine Aussage zur Präzision der Messungen getroffen werden.

\subsection{Zur Untersuchung der Moden des Reflexklystrons}
Die Moden des Reflexklystrons konnten auf einem Oszilloskopen sichtbar gemacht werden;
ihre graphische Nachbildung ist in Abbildung \ref{plt:moden} zusehen.
Obwohl offenbar der Anstieg der Amplitude mit dem Ansteig der Reflektorspannung zusammenhängt, kann aus genannter Abbildung nicht auf monotones Wachstum geschlossen werden.
Es scheinen die beiden Moden mit höchster Amplitude als Zwillingsmoden aufzutreten, deren Amplitude das Maximum  aller Modenamplituden ist.
Damit scheint auch ein Grenzwert der Modenamplitude zu existieren.
Um diese experimentelle Beobachtung und Hypothese weiter zu belegen oder zu widerlegen, sollte die Messung mit breiterem Reflektorspannungsspektrum wiederholt werden.

\subsection{Zur Untersuchung von Frequenz, Wellenlänge und Dämpfung}
Aus dem Vergleich der direkten Messung von der Hohlleiterfrequenz $f_\text{Messung}$ mit der theoretischen Frequenz $f_\text{Berechnung}$ aus Wellenlänge $\lambda_\text g$ und Länge der Hohlleiterbreitseite $a$ folgt eine Abweichung von $0.7\%$ und zeigt damit die Eignung der benutzen Methode.

Die Messung der Dämpfung zeigt starke Abweichung im Bereich von $90$--$100$\% des Theoriewertes.
Durch die durchgehend hohen Abweichungen, müssen systematischen Fehler in Betracht gezogen werden.
Die Abweichung könnte auf Probleme des Dämpfers, Probleme beim Ablesen der Dämpfereinstellungen oder auf den falschen Zusammenbau zurückführen, da diese Fehlerquellen die Messung und Berechnung der Frequenz mit guter Übereinstimmung nicht beeinflussen.

\subsection{Zur Untersuchung von stehenden Wellen}
Die Standing-Wave-Ratio $S$ liegt nach den drei vorgestellten Methoden zu drei verschiedenen Werten vor, die untereinander abweichen.
Da kein Idealwert vorliegt, kann die Qualität der Methoden nicht beziffert werden, und da die Abweichungen zwischen den Methoden groß ist, kann nur die Großenordnung erfasst werden.
Problematisch erscheint von den genannten Ergebnissen in Abschnitt \ref{sec:Auswertung_SWR} die Methode "SWR-Messer" für tief eingesetzte Nadeln. 
Durch die logarithmische Skalierung des Messinstruments ist präzises Ablesen erschwert, für eine Nadeltiefe von 9\,mm konnte kein brauchbares Ergebnis ermittelt werden.
Eine Wiederholung des Experimentes möge besonderes Augenmerk auf die Untersuchung von stehenden Wellen legen, die bei dieser Durchführung nur mangelhaft erfasst wurden. 