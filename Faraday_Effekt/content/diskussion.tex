\section{Diskussion}
\label{sec:Diskussion}
\subsection{Diskussion des Aufbaus}
Anstelle einer spektralen Halogen-Lampe kann ein geeignetes  monochromatisches Lasersystem im infraroten Bereich benutzt werden.
Dabei entfällt das Filtern der Wellenlänge.
Dieses Licht kann auf den Lichtzerhacker annähernd punktförmig fokussiert werden, sodass ein klares Rechteck-Signal entsteht.
Das Messsignal lässt sich durch einen Lock-In-Verstärker wegen des zusätzlichen Unterdrückens von frequenzfremden Rauschen besser auswerten als durch den hier benutzten Selektivverstärker.

Die Unsicherheiten in dem Magnetfeld könnten dadurch entstanden sein, dass die Hallsonde zum Messen des Feldes empfindlich auf Änderungen der Ausrichtung reagiert und die Hallsonde nicht fest eingespannt wurde.
Die Stromquelle versorgte den Magneten mit Strom, der über den gesamten Versuch nicht konstant ist, sondern leicht abnimmt.
Das Magnetfeld nahm also im Laufe der Messungen leicht ab, welches in der Rechnung nicht berücksichtigt wurde.
Durch diese mangelnde Präzision der Magnetfeldstärke schlagen sich in Gleichung \eqref{eqn:formel} mit linearer Abhängigkeit Messfehler nieder.

\subsection{Diskussion der Methode und Aussagekraft der Messergebnisse}
Durch starkes Rauschen ist das Finden des Endsignal-Minimums am Oszilloskop schwer. Es kann also nur mit geringer Genauigkeit eine Aussage über die effektive Masse getroffen werden.
Werden die Werte für die Differenzwinkel graphisch gegen $\lambda²$ aufgetragen, lässt sich erkennen, dass diese von der Geraden weit entfernt sind. Dies spricht für eine mangelhafte Messmethode. Allerdings streuen die Werte gleichmäßig um die Gerade, sodass die erhaltenen Werte der effektiven Elektronmasse recht nah am Literaturwert von $m_* = \SI{0.067}{m_e}$ \cite{Hurensohn3} sind.
Die Abweichungen betragen $13$ bzw. $31$ $\%$.
Diese sind bei der zweiten Probe größer, da durch das Absinken der magnetischen Feldstärke der systematische Fehler größer wird.
Eine Verbesserung wäre es, bei jedem Messvorgang der Proben auch das Magnetfeld erneut zu messen.

